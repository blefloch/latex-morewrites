% \iffalse meta-comment
%
%% File: morewrites.dtx Copyright (C) 2011-2015 Bruno Le Floch
%%
%% It may be distributed and/or modified under the conditions of the
%% LaTeX Project Public License (LPPL), either version 1.3c of this
%% license or (at your option) any later version.  The latest version
%% of this license is in the file
%%
%%    http://www.latex-project.org/lppl.txt
%%
%% -----------------------------------------------------------------------
%
%<*driver>
\RequirePackage{morewrites}
\documentclass[full]{l3doc}
\newwrite\foo \newwrite\foo \newwrite\foo \newwrite\foo
\newwrite\foo \newwrite\foo \newwrite\foo \newwrite\foo
\newwrite\foo \newwrite\foo \newwrite\foo \newwrite\foo
\newcommand{\proc}[1]{\texttt{#1}}
\begin{document}
  \DocInput{\jobname.dtx}
\end{document}
%</driver>
% \fi
%
% \title{The \textsf{morewrites} package: \\
%   Always room for a new \tn{write}\thanks{This
%     file has version number v0.4, last revised 2015/03/07.}}
% \author{Bruno Le Floch}
% \date{2015/03/07}
%
% \maketitle
% \tableofcontents
%
% \begin{documentation}
%
% \section{\pkg{morewrites} documentation}
%
% This \LaTeX{} package is a solution for the error
% \enquote{no room for a new \tn{write}}, which occurs when a document
% reserves too many streams to write data to various auxiliary files.
% It is in principle possible to rewrite other packages so that they
% are less greedy on resources, but that is often unpractical for the
% end-user.  Instead, \pkg{morewrites} hooks at the lowest level (\TeX{}
% primitives).  If I did my job correctly, you simply need to add the
% line |\usepackage{morewrites}| somewhere near the beginning of your
% \LaTeX{} file, and the \enquote{no room for a new \tn{write}} error
% should vanish.
%
% I have tried to make the code as robust as possible, but there may
% still be bugs lurking as this package has not been tested very
% thoroughly yet.  I thus encourage you to check that references are
% correct after loading that package: if they are correct without
% \pkg{morewrites}, but wrong with, please post a question on the
% \url{tex.stackexchange.com} question and answers website, send me
% a minimal file showing the problem, or log an issue on GitHub
% (\url{github.com/blefloch/latex-morewrites}).
%
% This package loads the \pkg{expl3} package, hence the \pkg{l3kernel}
% bundle needs to be up to date.
%
% \section{Commands defined or altered by \pkg{morewrites}}
%
% \begin{function}[added = 2014-07-26]{\morewritessetup}
%   \begin{syntax}
%     \cs{morewritessetup} \Arg{key--value list}
%   \end{syntax}
%   Set options.
% \end{function}
%
% \begin{function}[added = 2014-07-26, updated = 2015-03-07]{file}
%   \begin{syntax}
%     \cs{morewritessetup} |{| |file| |=| \meta{file name} |}|
%   \end{syntax}
%   Sets (globally) the name of the file which will be used by internal
%   processes of \pkg{morewrites}.  The file name is |\jobname.mw| by
%   default (technically, |\c_job_name_tl.mw|).
% \end{function}
%
% \begin{function}{\newwrite}
%   This command is altered by \pkg{morewrites}.  Since more than~$16$
%   write streams are now allowed, the corresponding restriction in
%   \tn{newwrite} becomes artificial and we remove it.
%   \begin{texnote}
%     The \tn{write} stream~$18$ is used by \TeX{} to access the
%     command line, and the stream~$16$ is often used for outputting
%     to the terminal, thus the revised \tn{newwrite} skips directly
%     from $15$ to~$20$ or so.  This might break some code that
%     expects \tn{newwrite} to allocate streams consecutively.
%   \end{texnote}
% \end{function}
%
% \begin{function}[updated = 2012-12-05]{\immediate}
%   This primitive is altered by \pkg{morewrites} in a complicated way.
% \end{function}
%
% \begin{function}{\openout, \write, \closeout}
%   These four primitives are altered by \pkg{morewrites} in complicated
%   ways, so that they accept stream numbers outside the normal range
%   $[0,15]$ and open/write/close files as appropriate.
% \end{function}
%
% \begin{function}{\shipout}
%   This primitive is altered by \pkg{morewrites} in a complicated way
%   which ensures that delayed \tn{write} commands, which are to be
%   performed at \tn{shipout} time, indeed are, and in the correct
%   order.
% \end{function}
%
% \section{Known deficiencies and open questions}
%
% See the bug tracker \url{github.com/blefloch/latex-morewrites/issues/}
% for a list of issues with \pkg{morewrites}.
%
% The package code is not very legible, and definitely uses too many
% |:D| control sequences, whose name means \enquote{do not use}.  I do
% not see a way to avoid using primitives in this package,
% since hooking into the primitives \tn{immediate}, \tn{write},
% \emph{etc.} requires having a very strong control on what every
% command does.  As a result, \emph{do not take this package as an
%   example of how to code with \pkg{expl3}; go and see Joseph Wright's
%   \pkg{siunitx} instead.}
%
% Things that need thought.
% \begin{itemize}
% \item Redefine the \LaTeX3 functions to use \pkg{morewrites} too?
% \item Should \tn{newwrite} be protected?
% \end{itemize}
%
% \end{documentation}
%
% \begin{implementation}
%
% \section{\pkg{morewrites} implementation}
%
%<*package>
%    \begin{macrocode}
\RequirePackage {expl3} [2012/08/14]
\RequirePackage {primargs} [2015/03/07]
\ProvidesExplPackage
  {morewrites} {2015/03/07} {0.4} {Always room for a new write}
%    \end{macrocode}
%
%    \begin{macrocode}
%<@@=morewrites>
%    \end{macrocode}
%
% \subsection{Overview of relevant \TeX{} facts}
%
% The aim of the \pkg{morewrites} package is to lift \TeX{}'s
% restriction of only having $16$ files open for writing at the same
% time.  We must thus patch the primitives \tn{openout}, \tn{write},
% and \tn{closeout} to accept numbers outside the range $[0,15]$,
% hence also \tn{immediate} to apply to the altered primitives, as
% well as \tn{special} for deferred actions.  Finally, the \TeX{}
% and \LaTeX{} macro \tn{newwrite} must allocate streams beyond~$15$.
% Let us review the syntax of the various commands we need to alter
% (see Chapter~24 of the \TeX{}book).
%
% We start with the three \enquote{actions}.
% \begin{quote}
%   \tn{openout} \meta{integer} \meta{equals} \meta{file name}
% \end{quote}
% \TeX{} searches the path for a file with a name given by
% \meta{file name}.  If found, this file is opened in the writing stream
% \meta{integer}, which must be a number in the range $[0,15]$.
% \begin{quote}
%   \tn{write} \meta{integer} \meta{filler} \meta{general text}
% \end{quote}
% \TeX{} expands the \meta{general text} as for an \texttt{x}-type
% expansion, with the caveat that macro parameter characters do not need
% to be doubled; converts the result to a string, and writes it in the
% writing stream \meta{integer}.  If the writing stream \meta{integer}
% is open (in particular it must be in the range $[0,15]$), then this
% writes to the corresponding file.  Otherwise, if the \meta{integer} is
% negative, the text is written to the log file, and a non-negative
% \meta{integer} writes to the terminal.  One exception: if the
% \meta{integer} is $18$, the text is sent to a shell to be run as shell
% code.
% \begin{quote}
%   \tn{closeout} \meta{integer}
% \end{quote}
% If the writing stream \meta{integer} is open, it is closed.
% Otherwise, if the \meta{integer} is not in the range $[0,15]$ an error
% may be raised, or nothing happens.
%
% By default, each one of those three \enquote{actions} are recorded in
% a whatsit node in the current list, and will be performed when the box
% containing the whatsit node is sent to the final \texttt{pdf},
% \emph{i.e.}, at \enquote{shipout} time.  In particular, the
% \meta{general text} for the \tn{write} primitive is expanded at
% shipout time.  This behaviour may be modified by putting
% \tn{immediate} before any of the three \enquote{actions} to force
% \TeX{} to perform the action immediately instead of recording it in a
% whatsit node.
%
% Since the \tn{openout}, \tn{write}, and \tn{closeout} primitives
% operate at \tn{shipout} time, we will have to hook into this primitive
% too.  It expects to be followed by a box specification, for instance
% \tn{box}\meta{integer} or \tn{hbox}\Arg{material to typeset}.
%
% Finally, the \tn{newwrite} macro expects one token as its argument,
% and defines this token (with \tn{chardef}) to be an integer
% corresponding to the first available writing stream.  We must extend
% it to let it allocate higher (virtual) write registers.
%
% All of the primitives above perform full expansion of all tokens when
% looking for their operands.  In most cases, only the \tn{meaning} of
% tokens encountered in this way matters.  Specifically,
% \begin{itemize}
% \item \meta{integer} denotes an integer in any form that \TeX{}
%   accepts as the right-hand side of a primitive integer assignment of
%   the form \tn{count}|0=|\meta{integer};
% \item \meta{equals} is an arbitrary (optional) number of explicit or
%   implicit space characters, an optional explicit equal sign of
%   category other, and further (optional) explicit or implicit space
%   characters;
% \item \meta{file name} is an arbitrary sequence of explicit or implicit
%   characters with arbitrary category codes (except active characters,
%   which are expanded before reaching \TeX{}'s mouth), ending either
%   with a space character (character code $32$, arbitrary non-active
%   category code, explicit or implicit), which is removed, or with a
%   non-expandable token, with some care needed for the case of a
%   \tn{notexpanded:} expandable token;
% \item \meta{filler} is an arbitrary combination of tokens whose
%   meaning is \tn{relax} or a character with category code $10$;
% \item \meta{general text} is formed of braced tokens, starting with an
%   explicit or implicit begin-group character, and ending with the
%   matching explicit end-group character (both with any character
%   code), with an equal number of explicit begin-group and end-group
%   characters in between: this is precisely the right-hand side of an
%   assignment of the form \tn{toks}|0=|\meta{general text}.
% \end{itemize}
%
% \subsection{Variants and helpers}
%
% \begin{macro}[aux]{\prop_gpop:NVNT, \tl_gput_right:Nv}
%   We need this function later.
%    \begin{macrocode}
\cs_generate_variant:Nn \prop_gpop:NnNT { NV }
\cs_generate_variant:Nn \tl_gput_right:Nn { Nv }
%    \end{macrocode}
% \end{macro}
%
% \begin{variable}[aux]{\l_@@_collect_next_int}
% \begin{macro}[aux]
%   {\@@_collect:x, \@@_collect_aux:Nn, \@@_collect_gput_right:N}
%   When encountering very large \tn{write} statements we may need to
%   collect many lines.  This can easily become an $O(n^2)$ task, and
%   here we make sure that it remains around $O(n\log n)$.  The
%   constant, unfortunately, is not small.
%    \begin{macrocode}
\int_new:N \l_@@_collect_next_int
\cs_new_protected:Npn \@@_collect:x #1
  {
    \tl_set:Nx \l_@@_internal_tl {#1}
    \@@_collect_aux:cf { l_@@_0_tl } { 1 }
  }
\cs_new_protected:Npn \@@_collect_aux:Nn #1#2
  {
    \int_compare:nNnT {#2} > \l_@@_collect_next_int
      {
        \tl_clear_new:N #1
        \int_set:Nn \l_@@_collect_next_int {#2}
      }
    \tl_if_empty:NTF #1
      { \tl_set_eq:NN #1 \l_@@_internal_tl }
      {
        \tl_put_left:No \l_@@_internal_tl {#1}
        \tl_clear:N #1
        \@@_collect_aux:cf { l_@@_#2_tl }
          { \int_eval:n { #2 + 1 } }
      }
  }
\cs_generate_variant:Nn \@@_collect_aux:Nn { cf }
\cs_new_protected:Npn \@@_collect_gput_right:N #1
  {
    \int_compare:nNnF \l_@@_collect_next_int = 0
      {
        \int_decr:N \l_@@_collect_next_int
        \tl_gput_right:Nv #1
          {
            l_@@_
            \int_use:N \l_@@_collect_next_int
            _tl
          }
        \@@_collect_gput_right:N #1
      }
  }
\cs_generate_variant:Nn \@@_collect_gput_right:N { c }
%    \end{macrocode}
% \end{macro}
% \end{variable}
%
% \subsection{Renaming primitives (again)}
%
% \begin{macro}[aux]
%   {
%     \@@_tex_immediate:w ,
%     \@@_tex_openout:w   ,
%     \@@_tex_write:w     ,
%     \@@_tex_closeout:w
%   }
%   Save the output-related primitives.
%    \begin{macrocode}
\cs_new_eq:NN \@@_tex_immediate:w \tex_immediate:D
\cs_new_eq:NN \@@_tex_openout:w   \tex_openout:D
\cs_new_eq:NN \@@_tex_write:w     \tex_write:D
\cs_new_eq:NN \@@_tex_closeout:w  \tex_closeout:D
%    \end{macrocode}
% \end{macro}
%
% \subsection{Variables}
%
% \begin{variable}[aux]{\@@_tmp:w}
%   Used for temporary definitions.
%    \begin{macrocode}
\cs_new_eq:NN \@@_tmp:w ?
%    \end{macrocode}
% \end{variable}
%
% \begin{variable}[aux]{\g_@@_later_int}
%   The integer \cs{g_@@_later_int} labels the various
%   non-immediate operations in the order in which they appear in the
%   source.  We can never reuse a number because there is no way to know
%   if a whatsit was recorded in a box register, which could be reused
%   in a shipped-out box:
%   \begin{quote}
%     \cs{vbox_set:Nn} \cs{l_my_box} \\
%     ~~|{| \cs{iow_shipout_x:Nn} \cs{c_term_iow} \Arg{text} |}|
%     \tn{shipout} \tn{copy} \cs{l_my_box}
%     \tn{shipout} \tn{copy} \cs{l_my_box}
%   \end{quote}
%   will print \meta{text} to the terminal twice.
%    \begin{macrocode}
\int_new:N \g_@@_later_int
%    \end{macrocode}
% \end{variable}
%
% \begin{variable}[aux]{\g_@@_iow_prop}
%   The property list \cs{g_@@_iow_prop} associates a file name to each
%   open stream.
%    \begin{macrocode}
\prop_new:N \g_@@_iow_prop
%    \end{macrocode}
% \end{variable}
%
% \begin{variable}[aux]{\g_@@_iow, \g_@@_ior}
%   The expansion that \tn{write} performs is impossible to emulate with
%   anything else than \tn{write}.  We will write on the stream
%   \cs{g_@@_iow} to the file \cs{g_@@_tmp_file_tl} and read back from
%   it in the stream \cs{g_@@_ior} for things to work properly.
%   Unfortunately, this means that the file is repeatedly opened and
%   closed, leaving a trace of that in the log.
%    \begin{macrocode}
\newwrite \g_@@_iow
\newread \g_@@_ior
%    \end{macrocode}
% \end{variable}
%
% \begin{variable}[aux]{\g_@@_reserved_iow_prop}
%   Some of the writing streams are already allocated when loading this
%   package, and we let the engine manage them.  This variable currently
%   maps these writing streams to themselves, but will eventually be used
%   with keys different from the values.
%    \begin{macrocode}
\prop_new:N \g_@@_reserved_iow_prop
\int_step_inline:nnnn {0} {1} { \g_@@_iow - 1 }
  { \prop_gput:Nnn \g_@@_reserved_iow_prop {#1} {#1} }
\prop_gput:Nnn \g_@@_reserved_iow_prop {18} {18}
%    \end{macrocode}
% \end{variable}
%
%^^A todo: is g_@@_stream_int needed?
% \begin{variable}[aux]{\g_@@_stream_int, \l_@@_stream_tl}
%   An integer holding the \meta{number} argument of various primitives,
%   namely a writing stream.  Also a token list occasionally containing
%   the same information, for use in |..._get| functions.
%    \begin{macrocode}
\int_new:N \g_@@_stream_int
\tl_new:N \l_@@_stream_tl
%    \end{macrocode}
% \end{variable}
%
% \begin{macro}[aux]{\s_@@}
%   A recognizable version of \cs{scan_stop:}.  This is inspired
%   by\footnote{Historically, this might have happened the other way
%     around, since the author of this package is also on the \LaTeX3
%     Team.} scan marks (see the \pkg{l3quark} module of \LaTeX3), but
%   note that we don't use \cs{__scan_new:N} directly, since it is
%   internal to \LaTeX3.
%    \begin{macrocode}
\cs_new_eq:NN \s_@@ \scan_stop:
%    \end{macrocode}
% \end{macro}
%
% \begin{variable}[aux]{\l_@@_internal_tl}
%   Temporary token list, used for scratch purposes.
%    \begin{macrocode}
\tl_new:N \l_@@_internal_tl
%    \end{macrocode}
% \end{variable}
%
% \subsection{Parsing}
%
% \begin{macro}[aux]{\@@_equals_file:N, \@@_equals_file_aux:}
%   Most of the parsing for primitive arguments is done using
%   \pkg{primargs}, except for one case we care about: after its
%   \meta{number} argument, the \tn{openout} primitive expects an
%   \meta{equals} (optional spaces and |=|) and a \meta{file name}.
%    \begin{macrocode}
\cs_new_protected:Npn \@@_equals_file:N #1
  {
    \group_begin:
      \tex_aftergroup:D #1
      \primargs_remove_equals:N \@@_equals_file_aux:
  }
\cs_new_protected_nopar:Npn \@@_equals_file_aux:
  { \primargs_get_file_name:N \group_end: }
%    \end{macrocode}
% \end{macro}
%
% \subsection{Auxiliary file}
%
% \begin{variable}[aux]{\g_@@_tmp_file_tl, \g_@@_tmp_file_bool}
%   Temporary file used to do the correct expansion for each \tn{write}.
%   Boolean indicating whether that file has been checked to be empty:
%   the \pkg{morewrites} package will only use the file if it is indeed
%   empty, so as to avoid clobbering user data.
%    \begin{macrocode}
\tl_new:N \g_@@_tmp_file_tl
\bool_new:N \g_@@_tmp_file_bool
%    \end{macrocode}
% \end{variable}
%
% \begin{macro}{\@@_set_file:n}
%   Sets \cs{g_@@_tmp_file_tl} to the given value (|\c_job_name_tl.mw|
%   by default), and sets a boolean to indicate that the file has not
%   been tested for emptyness.
%    \begin{macrocode}
\cs_new_protected:Npn \@@_set_file:n #1
  {
    \tl_gset:Nn \g_@@_tmp_file_tl {#1}
    \bool_gset_false:N \g_@@_tmp_file_bool
  }
%    \end{macrocode}
% \end{macro}
%
% \begin{macro}[aux]{\@@_empty_file:n}
%   Empties the file \cs{g_@@_tmp_file_tl} by opening it and closing it
%   right away.  This ensures that the file used by \pkg{morewrites} is
%   left empty (and we do this every time the auxiliary file is used,
%   in case that file is changed by \cs{morewritessetup} mid-document).
%    \begin{macrocode}
\cs_new_protected:Npn \@@_empty_file:n #1
  {
    \@@_tex_immediate:w \@@_tex_openout:w
      \g_@@_iow = #1 \scan_stop:
    \@@_tex_immediate:w \@@_tex_closeout:w
      \g_@@_iow
  }
%    \end{macrocode}
% \end{macro}
%
% \begin{macro}[aux,TF]{\@@_if_file_trivial:n}
%   True if the file does not exist, or if it is empty.
%   Only the TF variant is defined.  We set
%   \cs{@@_tmp:w} to \cs{prg_return_true:} or \cs{prg_return_false:}
%   within the group and use it after cleaning up.  The first
%   \texttt{eof} test is \texttt{true} if the file does not exist.
%   Then we read one line, the second \texttt{eof} test is
%   \texttt{true} if the file was empty (it is \texttt{false} if
%   the file contained anything, even a single space).
%    \begin{macrocode}
\prg_new_conditional:Npnn \@@_if_file_trivial:n #1 { TF }
  {
    \group_begin:
      \tex_openin:D \g_@@_ior = #1 \scan_stop:
      \if_eof:w \g_@@_ior
        \cs_gset_eq:NN \@@_tmp:w \prg_return_true:
      \else:
        \int_set_eq:NN \tex_endlinechar:D \c_minus_one
        \etex_readline:D \g_@@_ior to \l_@@_internal_tl
        \if_eof:w \g_@@_ior
          \cs_gset_eq:NN \@@_tmp:w \prg_return_true:
        \else:
          \cs_gset_eq:NN \@@_tmp:w \prg_return_false:
        \fi:
      \fi:
      \tex_closein:D \g_@@_ior
    \group_end:
    \@@_tmp:w
  }
%    \end{macrocode}
% \end{macro}
%
% \begin{macro}[aux]{\@@_chk_file:}
%   Check that the file \cs{g_@@_tmp_file_tl} does not exist or is blank.
%   If not, try the file name obtained by adding |.mw|.
%   This avoids clobbering files that the user would not want to lose.
%    \begin{macrocode}
\cs_new_protected_nopar:Npn \@@_chk_file:
  {
    \@@_if_file_trivial:nTF { \g_@@_tmp_file_tl }
      { \bool_gset_true:N \g_@@_tmp_file_bool }
      {
        \msg_warning:nnxx { morewrites } { file-exists }
          { \g_@@_tmp_file_tl }
          { \g_@@_tmp_file_tl .mw }
        \tl_gput_right:Nn \g_@@_tmp_file_tl { .mw }
        \@@_chk_file:
      }
  }
\msg_new:nnnn { morewrites } { file-exists }
  { File~'#1'~exists,~using~'#2'~instead. }
  {
    The~file~`#1'~exists~and~was~not~created~by~this~version~of~the~
    `morewrites'~package.~Please~move~or~delete~that~file,~or~provide~
    another~file~name~by~adding
    \\ \\
    \iow_indent:n { \iow_char:N\\morewritessetup~{~file~=~other-name~} }
    \\ \\
    to~your~source~file.~In~the~meantime,~the~file~`#2'~will~be~used.
  }
%    \end{macrocode}
% \end{macro}
%
% \subsection{Immediate (writing)}
%
% In the context of immediate writing, we can store the text
% in a token list, and only write it at the corresponding
% \tn{closeout} command. We keep track of a property list,
% \cs{g_@@_iow_prop}, of the writes which are
% open (from the point of view of the user), with the
% corresponding file name.
%
% \subsubsection{What follows \tn{immediate}}
%
% \begin{macro}[aux, updated = 2012-12-05]{\@@_immediate:w}
% \begin{macro}[aux]
%   {\@@_immediate_auxii:, \@@_immediate_auxiii:N, \@@_immediate_auxiv:NN}
% \begin{macro}[aux, EXP]{\@@_immediate_auxv:ww}
%   This is a little bit subtle: \TeX{}'s \tn{immediate} primitive
%   raises a flag which is cancelled once \TeX{} sees a non-expandable
%   token.  We use \pkg{primargs}'s \texttt{read_x_token} function to
%   fully expand in the \TeX{} way, then test for \tn{openout},
%   \tn{write}, or \tn{closeout}.  We don't test for the primitives
%   themselves, but rather for a recognizable marker, \cs{s_@@}, equal
%   to \tn{relax}.  If present, replace |morewrites| by
%   |morewrites_now| in the csname of the second token after it
%   (it turns out that this is the correct structure).  If absent, what
%   follows may be a command that should not appear after
%   \tn{immediate}, but it may also be a non-\TeX\ primitive such as
%   \tn{pdfobj} that the \pkg{morewrite} does not know about; hence we
%   must still call the primitive \tn{immediate}.
%    \begin{macrocode}
\cs_new_protected_nopar:Npn \@@_immediate:w
  { \primargs_read_x_token:N \@@_immediate_auxii: }
\cs_new_protected_nopar:Npn \@@_immediate_auxii:
  {
    \token_if_eq_meaning:NNTF \g_primargs_token \s_@@
      { \@@_immediate_auxiii:N }
      { \@@_tex_immediate:w }
  }
\cs_new_protected:Npn \@@_immediate_auxiii:N #1
  {
    \tl_if_eq:nnTF { #1 } { \s_@@ }
      { \@@_immediate_auxiv:NN }
      { #1 }
  }
\cs_new_protected:Npn \@@_immediate_auxiv:NN #1 #2
  {
    \exp_args:Nc #1
      {
        \exp_after:wN \@@_immediate_auxv:ww
        \token_to_str:N #2
      }
  }
\use:x
  {
    \cs_new:Npn \exp_not:N \@@_immediate_auxv:ww
        ##1 \tl_to_str:n { @@ _ }
        ##2 \token_to_str:N _
      { @@_##2_now_ }
  }
%    \end{macrocode}
% \end{macro}
% \end{macro}
% \end{macro}
%
% \subsubsection{Immediate closeout}
%
% \begin{macro}[aux]{\@@_closeout_now_test:n}
%   When the user requests to close a stream, we look in
%   \cs{g_@@_reserved_iow_prop} to see if it is a reserved
%   stream: in this case, we simply use the primitive.
%    \begin{macrocode}
\cs_new_protected:Npn \@@_closeout_now_test:n #1
  {
    \int_gset:Nn \g_@@_stream_int {#1}
    \prop_get:NnNTF \g_@@_reserved_iow_prop {#1}
        \l_@@_stream_tl
      {
        \@@_tex_immediate:w \@@_tex_closeout:w
          \l_@@_stream_tl
      }
      { \@@_closeout_now_aux: }
  }
%    \end{macrocode}
% \end{macro}
%
% \begin{macro}[aux]{\@@_closeout_now_aux:}
%   We then look in \cs{g_@@_iow_prop} to find the file name
%   corresponding to that stream number.  If the stream does not appear
%   as a key in the property list, then it was not open yet, and we do
%   nothing.  Otherwise, the key is removed, and we write the collected
%   material to the file.
%    \begin{macrocode}
\cs_new_protected_nopar:Npn \@@_closeout_now_aux:
  {
    \prop_gpop:NVNT \g_@@_iow_prop \g_@@_stream_int
      \l_@@_internal_tl
      {
        \@@_closeout_now:nn
          { \g_@@_stream_int } { \l_@@_internal_tl }
      }
  }
%    \end{macrocode}
% \end{macro}
%
% \begin{macro}[aux]{\@@_closeout_now:nn}
%   The code to write the material collected so far for a given output
%   \meta{stream} is in the token list
%   \cs{g__morewrites_iow_\meta{stream}_tl}.
%   We do this writing in the actual stream \cs{g_@@_iow}, briefly
%   opened and closed on the file |#2|.
%    \begin{macrocode}
\cs_new_protected:Npn \@@_closeout_now:nn #1#2
  {
    \@@_tex_immediate:w \@@_tex_openout:w
      \g_@@_iow = #2 \scan_stop:
    \group_begin:
      \int_set_eq:NN \tex_newlinechar:D \c_minus_one
      \tl_use:c { g_@@_iow_ \int_eval:n {#1} _tl }
      \tl_gclear:c { g_@@_iow_ \int_eval:n {#1} _tl }
    \group_end:
    \@@_tex_immediate:w \@@_tex_closeout:w
      \g_@@_iow
  }
%    \end{macrocode}
% \end{macro}
%
% \subsubsection{Immediate openout}
%
% \begin{macro}[aux]{\@@_openout_now_test:n}
%   Read the stream number.  If it is one of the reserved streams, we
%   use the primitive.  Otherwise, parse an optional equal sign,
%   followed by the file name.
%    \begin{macrocode}
\cs_new_protected:Npn \@@_openout_now_test:n #1
  {
    \int_gset:Nn \g_@@_stream_int {#1}
    \prop_get:NnNTF \g_@@_reserved_iow_prop {#1}
        \l_@@_stream_tl
      {
        \@@_tex_immediate:w \@@_tex_openout:w
          \l_@@_stream_tl
      }
      {
        \@@_equals_file:N
          \@@_openout_now_aux:n
      }
  }
%    \end{macrocode}
% \end{macro}
%
% \begin{macro}[aux]{\@@_openout_now_aux:n}
%   When the user requests to open a stream, it might already be open,
%   with another file as its destination. We thus need to first close
%   the stream, writing all that we collected so far to that other
%   file. This has no effect if the stream was not open yet.
%
%   We then put the stream and its associated file name in the property
%   list, and empty/create the corresponding token list.
%   Finally, empty the target file as \TeX{}'s \tn{openout} would do.
%    \begin{macrocode}
\cs_new_protected:Npn \@@_openout_now_aux:n #1
  {
    \@@_closeout_now_aux:
    \prop_gput:NVn \g_@@_iow_prop \g_@@_stream_int {#1}
    \tl_gclear_new:c
      { g_@@_iow_ \int_use:N \g_@@_stream_int _tl }
    \@@_empty_file:n {#1}
  }
%    \end{macrocode}
% \end{macro}
%
% \subsubsection{Immediate write}
%
% \begin{macro}[aux]{\@@_write_now_test:n}
%   Read the stream number.  If it is one of the reserved streams, we
%   use the primitive.  Otherwise, parse the text.
%    \begin{macrocode}
\cs_new_protected:Npn \@@_write_now_test:n #1
  {
    \int_gset:Nn \g_@@_stream_int {#1}
    \prop_get:NnNTF \g_@@_reserved_iow_prop {#1}
        \l_@@_stream_tl
      {
        \@@_tex_immediate:w \@@_tex_write:w
          \l_@@_stream_tl
      }
      { \primargs_get_general_text:N \@@_write_now_aux:n }
  }
%    \end{macrocode}
% \end{macro}
%
% \begin{macro}[aux]{\@@_write_now_aux:n}
%   Test whether the stream is allocated or not.
%    \begin{macrocode}
\cs_new_protected_nopar:Npn \@@_write_now_aux:n
  {
    \prop_get:NVNTF \g_@@_iow_prop \g_@@_stream_int
      \l_@@_internal_tl
      { \@@_write_now_open:n }
      { \@@_write_now_closed:n }
  }
%    \end{macrocode}
% \end{macro}
%
% \begin{macro}[aux]{\@@_write_now_closed:n}
%   If the stream \cs{g_@@_stream_int} is not allocated, then write
%   either to the terminal or only to the log file, depending on the
%   sign.
%    \begin{macrocode}
\cs_new_protected_nopar:Npn \@@_write_now_closed:n
  {
    \@@_tex_immediate:w \@@_tex_write:w
      \int_compare:nNnTF \g_@@_stream_int < \c_zero { -1 } { 16 }
  }
%    \end{macrocode}
% \end{macro}
%
% \begin{macro}[aux]{\@@_write_now_open:n}
% \begin{macro}[aux]{\@@_write_now_loop:}
%   Only \tn{write} itself can emulate how \tn{write} expands tokens,
%   because |#| don't have to be doubled, and because the
%   \tn{newlinechar} has to be changed to new lines.  Hence, we start by
%   writing |#1| to a file (after making sure we are allowed to alter
%   it), yielding some lines.  The lines are then
%   read one at a time using \eTeX{}'s \tn{readline} with
%   \tn{endlinechar} set to $-1$ to avoid spurious characters.  Each
%   line becomes a \tn{immediate} \tn{write} statement added to the
%   token list \cs{g_@@_iow_\meta{stream}_tl}.  This token list will be
%   called when it is time to actually write to the file.  At that time,
%   \tn{newlinechar} will be $-1$, so that writing each line will
%   produce no extra line.
%    \begin{macrocode}
\cs_new_protected:Npn \@@_write_now_open:n #1
  {
    \bool_if:NF \g_@@_tmp_file_bool { \@@_chk_file: }
    \@@_tex_immediate:w \@@_tex_openout:w
      \g_@@_iow = \g_@@_tmp_file_tl \scan_stop:
    \@@_tex_immediate:w \@@_tex_write:w
      \g_@@_iow {#1}
    \@@_tex_immediate:w \@@_tex_closeout:w
      \g_@@_iow
    \group_begin:
      \int_set_eq:NN \tex_endlinechar:D \c_minus_one
      \tex_openin:D \g_@@_ior = \g_@@_tmp_file_tl \scan_stop:
      \@@_write_now_loop:
      \tex_closein:D \g_@@_ior
      \@@_collect_gput_right:c
        { g_@@_iow_ \int_use:N \g_@@_stream_int _tl }
    \group_end:
    \@@_empty_file:n { \g_@@_tmp_file_tl }
  }
\cs_new_protected_nopar:Npn \@@_write_now_loop:
  {
    \etex_readline:D \g_@@_ior to \l_@@_internal_tl
    \ior_if_eof:NF \g_@@_ior
      {
        \@@_collect:x
          {
            \@@_tex_immediate:w \@@_tex_write:w
              \g_@@_iow { \l_@@_internal_tl }
          }
        \@@_write_now_loop:
      }
  }
%    \end{macrocode}
% \end{macro}
% \end{macro}
%
%
% \subsection{Non-immediate writing}
%
% This is trickier, because the expansion of the text for a
% non-immediate \tn{write} takes place immediately after the page
% containing it is shipped out.  We store each non-immediate
% \tn{openout}, \tn{write}, or \tn{closeout} without expansion in
% separate token lists \cs{g_@@_later_\meta{stream}_tl} to be used
% later, and instead write |`(|\meta{stream}|)| to a file (including the
% strange delimiters).  After each shipout, we can read the file to see
% which output operations we need to perform, and in what order.
%
% \subsubsection{Replacement for primitives}
%
% \begin{macro}[aux]{\@@_later:n}
%   Store the action to be done at shipout in a token list, and
%   non-immediately write the label \cs{g_@@_later_int} of the
%   output operation to the temporary file.  Here, |#1| holds an
%   assignment similar to the lines above it, and |#2| holds the
%   relevant immediate action to be performed after shipout.
%    \begin{macrocode}
\cs_new_protected:Npn \@@_later:n #1
  {
    \int_gincr:N \g_@@_later_int
    \tl_const:cx
      {
        c_@@_later_
        \int_use:N \g_@@_later_int
        _tl
      }
      {
        \int_gset:Nn \exp_not:N \g_@@_stream_int
          { \exp_not:V \g_@@_stream_int }
        \exp_not:n {#1}
      }
    \exp_args:NNx \@@_tex_write:w \g_@@_iow
      { `( \int_use:N \g_@@_later_int ) }
  }
%    \end{macrocode}
% \end{macro}
%
% \begin{macro}[aux]{\@@_openout:w}
% \begin{macro}[aux]{\@@_openout_test:n, \@@_openout_aux:n}
%   \tn{openout} tests if the number to come is among reserved streams.
%   If it is, use the primitive, otherwise, parse a file name.
%    \begin{macrocode}
\cs_new_protected_nopar:Npn \@@_openout:w
  { \s_@@ \primargs_get_number:N \@@_openout_test:n }
\cs_new_protected:Npn \@@_openout_test:n #1
  {
    \int_gset:Nn \g_@@_stream_int {#1}
    \prop_get:NnNTF \g_@@_reserved_iow_prop {#1}
        \l_@@_stream_tl
      { \@@_tex_openout:w \l_@@_stream_tl }
      { \@@_equals_file:N \@@_openout_aux:n }
  }
\cs_new_protected:Npn \@@_openout_aux:n #1
  { \@@_later:n { \@@_openout_now_aux:n {#1} } }
%    \end{macrocode}
% \end{macro}
% \end{macro}
%
% \begin{macro}[aux]{\@@_write:w}
% \begin{macro}[aux]{\@@_write_test:n, \@@_write_aux:n}
%   Same idea for \tn{write}, except that we parse a text.
%    \begin{macrocode}
\cs_new_protected_nopar:Npn \@@_write:w
  { \s_@@ \primargs_get_number:N \@@_write_test:n }
\cs_new_protected:Npn \@@_write_test:n #1
  {
    \int_gset:Nn \g_@@_stream_int {#1}
    \prop_get:NnNTF \g_@@_reserved_iow_prop {#1}
        \l_@@_stream_tl
      { \@@_tex_write:w \l_@@_stream_tl }
      { \primargs_get_general_text:N \@@_write_aux:n }
  }
\cs_new_protected:Npn \@@_write_aux:n #1
  { \@@_later:n { \@@_write_now_aux:n {#1} } }
%    \end{macrocode}
% \end{macro}
% \end{macro}
%
% \begin{macro}[aux]{\@@_closeout:w}
% \begin{macro}[aux]{\@@_closeout_test:n, \@@_closeout_aux:}
%   Same idea for \tn{closeout}, and we don't need to parse
%   anything else than the number.
%    \begin{macrocode}
\cs_new_protected_nopar:Npn \@@_closeout:w
  { \s_@@ \primargs_get_number:N \@@_closeout_test:n }
\cs_new_protected:Npn \@@_closeout_test:n #1
  {
    \int_gset:Nn \g_@@_stream_int {#1}
    \prop_get:NnNTF \g_@@_reserved_iow_prop {#1}
        \l_@@_stream_tl
      { \@@_tex_closeout:w \l_@@_stream_tl }
      { \@@_closeout_aux: }
  }
\cs_new_protected_nopar:Npn \@@_closeout_aux:
  { \@@_later:n { \@@_closeout_now_aux: } }
%    \end{macrocode}
% \end{macro}
% \end{macro}
%
% \subsubsection{Shipout business}
%
% In this section, we hook into the \tn{shipout} primitive, and redefine
% it to first build a box with the material to ship out, then perform
% \begin{quote}
%   \cs{@@_before_shipout:} \\
%   \meta{primitive shipout} \meta{collected box} \\
%   \cs{@@_after_shipout:}
% \end{quote}
% This is correct even if the values of the \tn{newlinechar} is changed
% within the user code which builds the shipped out box, because the
% value that \TeX{} uses is the value in effect immediately after
% \tn{shipout}.
%
% \begin{macro}[aux]{\@@_before_shipout:}
%   Immediately before the shipout, we must open the writing stream
%   \cs{g_@@_iow} (after making sure we are allowed to alter the
%   auxiliary file).  Each delayed output operation has been replaced by
%   \tn{write} \cs{g_@@_iow} |{`(|\meta{operation number}|)}|.  The
%   delimiters we chose to put around numbers must be at least two
%   distinct characters on the left (then \cs{tex_newlinechar:D} cannot
%   be equal to the delimiter), and at least one non-digit character on
%   the right.
%    \begin{macrocode}
\cs_new_protected_nopar:Npn \@@_before_shipout:
  {
    \bool_if:NF \g_@@_tmp_file_bool { \@@_chk_file: }
    \@@_tex_immediate:w \@@_tex_openout:w
      \g_@@_iow = \g_@@_tmp_file_tl \scan_stop:
  }
%    \end{macrocode}
% \end{macro}
%
% \begin{macro}[aux]{\@@_after_shipout:}
% \begin{macro}[aux, rEXP]{\@@_after_shipout_loop:ww}
%   Immediately after all the \tn{write}s are performed, close the file,
%   then read the file with \tn{endlinechar} set to
%   \tn{newlinechar}\footnote{Note that the \tn{newlinechar} used by
%     \tn{write}s at \tn{shipout} time are those in effect when the page
%     is shipped out, \emph{i.e.}, just after the closing brace of the
%     \tn{shipout} construction, which is exactly where we have added
%     this hook.} to get exactly the original characters that have been
%   written, possibly with extra characters between |`(|\ldots{}|)|
%   groups.  The file is then read with all the appropriate category
%   codes set up (no other character can appear in the file).  The
%   looping auxiliary \cs{@@_after_shipout_loop:ww} extract the
%   \meta{operation} numbers from the file, and makes a token list out
%   of those.  This token list is then used in a mapping function to
%   perform the appropriate \tn{write} operations.  Note that those
%   operations may reuse the file, so we have to fully parse the file
%   before moving on.
%    \begin{macrocode}
\cs_new_protected_nopar:Npn \@@_after_shipout:
  {
    \@@_tex_immediate:w \@@_tex_closeout:w
      \g_@@_iow
    \group_begin:
      \int_set_eq:NN \tex_endlinechar:D \tex_newlinechar:D
      \char_set_catcode_other:n { \tex_endlinechar:D }
      \tl_map_inline:nn { `(0123456789) }
        { \char_set_catcode_other:n {`##1} }
      \etex_everyeof:D { `() \exp_not:N }
      \tl_set:Nx \l_@@_internal_tl
        {
          \exp_after:wN \@@_after_shipout_loop:ww
          \tex_input:D \g_@@_tmp_file_tl \c_space_tl
        }
      \@@_empty_file:n { \g_@@_tmp_file_tl }
      \exp_args:NNo
    \group_end:
    \tl_map_inline:nn { \l_@@_internal_tl }
      { \tl_use:c { c_@@_later_ ##1 _tl } }
  }
\cs_new:Npn \@@_after_shipout_loop:ww #1 `( #2 )
  {
    \tl_if_empty:nF {#2}
      {
        {#2}
        \@@_after_shipout_loop:ww
      }
  }
%    \end{macrocode}
% \end{macro}
% \end{macro}
%
% \begin{variable}[aux]{\g_@@_group_level_int, \g_@@_shipout_box}
%   Store the page to be shipped out in a box.  Store the group level
%   when \tn{shipout} is called: this is used to distinguish between
%   explicit boxes and box registers.
%    \begin{macrocode}
\int_new:N \g_@@_group_level_int
\box_new:N \g_@@_shipout_box
%    \end{macrocode}
% \end{variable}
%
% \begin{macro}[aux]{\@@_shipout:w}
% \begin{macro}[aux]{\@@_shipout_i:, \@@_shipout_ii:}
%   Store the group level
%    \begin{macrocode}
\cs_new_protected_nopar:Npn \@@_shipout:w
  {
    \int_gset_eq:NN \g_@@_group_level_int \etex_currentgrouplevel:D
    \tex_afterassignment:D \@@_shipout_i:
    \tex_global:D \tex_setbox:D \g_@@_shipout_box
  }
\cs_new_protected_nopar:Npn \@@_shipout_i:
  {
    \int_compare:nNnTF { \g_@@_group_level_int }
                     = { \etex_currentgrouplevel:D }
      { \@@_shipout_ii: }
      { \tex_aftergroup:D \@@_shipout_ii: }
  }
\cs_new_protected_nopar:Npn \@@_shipout_ii:
  {
    \@@_before_shipout:
    \@@_tex_shipout:w \tex_box:D \g_@@_shipout_box
    \@@_after_shipout:
  }
%    \end{macrocode}
% \end{macro}
% \end{macro}
%
% \begin{macro}[aux]{\shipout, \@@_tex_shipout:w}
%   The task is now to locate the shipout primitive, which may have been
%   renamed and hooked into by many different packages loaded before
%   \pkg{morewrites}.  Any of those control sequences which are equal to
%   the primitive are redefined to do \cs{@@_shipout:w} instead.  If the
%   primitive is not located at all, the fallback is to hook into the
%   control sequence \tn{shipout}.
%    \begin{macrocode}
\cs_gset_protected:Npn \@@_tmp:w #1
  {
    \cs_if_exist:NF \@@_tex_shipout:w
      { \cs_new_eq:NN \@@_tex_shipout:w #1 }
    \cs_gset_eq:NN #1 \@@_shipout:w
  }
\tl_map_inline:nn
  {
    \xyrealshipout@
    \org@shipout
    \PDFSYNCship@ut@ld
    \CROP@shipout
    \@soORI
    \tex_shipout:D
    \zwpl@Hship
    \o@shipout@TP
    \LL@shipout
    \Shipout
    \GXTorg@shipout
    \AtBegShi@OrgShipout
    \AtBeginShipoutOriginalShipout
  }
  {
    \str_if_eq_x:nnT
      { \cs_meaning:N #1 }
      { \token_to_str:N \shipout }
      { \@@_tmp:w #1 }
  }
\@@_tmp:w \shipout
%    \end{macrocode}
% \end{macro}
%
% \subsubsection{Hook at the very end}
%
% \begin{macro}[aux]{\@@_close_all:}
%   At the end of the document, close all the files.
%    \begin{macrocode}
\cs_new_protected_nopar:Npn \@@_close_all:
  {
    \prop_map_function:NN \g_@@_iow_prop
      \@@_closeout_now:nn
    \prop_gclear:N \g_@@_iow_prop
  }
%    \end{macrocode}
% \end{macro}
%
% \begin{macro}[aux]{\@@_close_all_at_end:nw}
%   At the end of the run, we try very hard to put some material at the
%   \tn{@@end}, just in case some other very late code writes to files
%   that are not yet closed.  This is tried at most $5$~times, to avoid
%   infinite loops in case two packages compete for that last place.
%   The definition is performed using \cs{@@_tmp:w} because the
%   \LaTeXe{} command \tn{@@end} would otherwise be messed up by
%   \pkg{l3docstrip}.
%    \begin{macrocode}
\cs_gset_protected:Npn \@@_tmp:w #1
  {
    \cs_new_protected:Npn \@@_close_all_at_end:nw ##1##2 #1
      {
        \int_compare:nNnTF {##1} > \c_zero
          { ##2 \@@_close_all_at_end:nw { ##1 - 1 } }
          { \@@_close_all: ##2 }
        #1
      }
  }
\exp_args:Nc \@@_tmp:w { @ @ end }
\AtEndDocument { \@@_close_all_at_end:nw { 5 } }
%    \end{macrocode}
% \end{macro}
%
% \subsection{Redefining commands}
%
% \subsubsection{Modified \tn{newwrite}}
%
% \begin{variable}[aux]{\g_@@_alloc_int}
%   The counter that \LaTeXe{} uses to allocate \tn{write} registers.
%    \begin{macrocode}
\tex_countdef:D \g_@@_alloc_int 17 \scan_stop:
%    \end{macrocode}
% \end{variable}
%
% \begin{macro}[aux]{\newwrite}
%   We need to allow \tn{newwrite} to allocate more than $16$ writes,
%   but beware that $18$ is reserved, and that packages might expect
%   $16$ or $17$ to write to the terminal. So instead skip until $20$,
%   to be on the safe side.  This really ought to be \tn{protected}, but
%   none of the formats do that.
%    \begin{macrocode}
\cs_new:Npn \@@_newwrite:N #1
  {
    \int_gincr:N \g_@@_alloc_int
    \if_int_compare:w \g_@@_alloc_int = \c_sixteen
      \int_gset:Nn \g_@@_alloc_int { 20 }
    \fi:
    \int_set_eq:NN \allocationnumber \g_@@_alloc_int
    \cs_undefine:N #1
    \int_const:Nn #1 { \allocationnumber }
    \wlog
      {
        \token_to_str:N #1
        = \token_to_str:N \write \int_use:N \allocationnumber
      }
  }
%    \end{macrocode}
% \end{macro}
%
% \subsection{User commands and keys}
%
% \begin{macro}[added = 2014-07-26]{\morewritessetup}
%   Set whatever keys the user passes to \cs{morewritessetup}.
%    \begin{macrocode}
\cs_new_protected:Npn \morewritessetup #1
  { \keys_set:nn { @@ } {#1} }
%    \end{macrocode}
% \end{macro}
%
% \begin{macro}[added = 2014-07-26]{file}
%   Because of our use of |.initial:n|, this code must appear after
%   \cs{@@_set_file:n} is defined.
%    \begin{macrocode}
\keys_define:nn { @@ }
  {
    file .code:n = \@@_set_file:n {#1} ,
    file .initial:n = \c_job_name_tl .mw
  }
%    \end{macrocode}
% \end{macro}
%
% \begin{macro}[updated = 2012-12-05]{\immediate}
% \begin{macro}{\openout, \write, \closeout, \newwrite}
%   \tn{shipout} has been redefined earlier.
%    \begin{macrocode}
\cs_gset_eq:NN \immediate \@@_immediate:w
\cs_gset_eq:NN \openout   \@@_openout:w
\cs_gset_eq:NN \write     \@@_write:w
\cs_gset_eq:NN \closeout  \@@_closeout:w
\cs_gset_eq:NN \newwrite  \@@_newwrite:N
%    \end{macrocode}
% \end{macro}
% \end{macro}
%
%</package>
%
% \end{implementation}
%
% \clearpage
% \PrintIndex
